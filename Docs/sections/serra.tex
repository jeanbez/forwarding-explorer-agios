As previously discussed, the Prediction Module requires information about the user's storage device to perform scheduling algorithm selection. Additionally, other parts of the library implementation also use this information. This information is provided to AGIOS through two .h files, generated by SeRRa. The tool and information on how to use it can be obtained at \url{http://www.inf.ufrgs.br/~fzboito/serra.html}.

\section{Access times file}

The first .h file, \emph{access\_times.h}, define two matrices - \emph{writing\_times} and \emph{reading\_times} - and three constants to help obtaining information from them.

\begin{lstlisting}[language=C]
static unsigned int writing_times[][2] = 
		      { { 1024, 6482209 },
			{ 4096, 2028673 },
			{ 8192, 606673 },
			{ 12288, 406417 },
			{ 16384, 1040055 } };

static unsigned int reading_times[][2] = 
		      { { 1024, 27779 },
			{ 4096, 9596 },
			{ 8192, 9927 },
			{ 12288, 11527 },
			{ 16384, 13828 } };

#define MAX_BENCHMARK_SIZE 16 //in KB

#define NUMBER_OF_TESTS 5

#define BENCHMARK_STEP 4 //in KB
\end{lstlisting}

We show above an example of these definitions. The two data structures contain pairs \{ request size, access time in nanoseconds \}. Request file sizes should be chosen as to represent most of requests that will be scheduled. Moreover, sizes a few times larger (at least $16$ times larger, but as much as possible) should also be included in order to help guiding aggregation decisions. Notice, however, that a regular step between consecutive pairs' request sizes must be respected. 

These times provided in \emph{access\_times.h} are used by the aIOLi scheduling algorithm (to quantum assignment) and by the Prediction Module (to maintain the $\alpha$ factor and test against predicted aggregations). If you are using AGIOS without predicted request aggregations and with another scheduling algorithm, you can provide any values (the example file that comes within the library package is OK).

\section{Access times ratios file}

The second file, \emph{access\_times\_ratios.h} defines two matrices - \emph{writing\_ratios} and \emph{reading\_ratios} - and two constants to help obtain information from these matrices.

\begin{lstlisting}[language=C]
static float writing_ratios[][2] = { { 8192, 11.81 },
			             { 16384, 8.78 },
				     { 24576, 8.99 },
				     { 32768, 6.00 },
				     { 40960, 8.38 },
				     { 49152, 7.37 },
				     { 57344, 6.51 },
				     { 65536, 2.85 } };

static float reading_ratios[][2] = { { 8192, 36.88 },
				     { 16384, 31.71 },
				     { 24576, 25.17 },
				     { 32768, 21.81 },
				     { 40960, 18.93 },
				     { 49152, 16.85 },
				     { 57344, 15.07 },
				     { 65536, 11.80 } };

#define NUMBER_OF_RATIOS 8

#define RATIOS_STEP 8 //in KB
\end{lstlisting} 

The data structures defined in the example above are composed of pairs \{ request size, sequential to random throughput ratio \}. The request sizes must represent most of the possible requests, with no need to consider larger requests due to aggregations.

These values are used by the Prediction Module for scheduling algorithm selection. If you are using AGIOS without automatic algorithm selection, you can provide any values (the example file that comes within the library package is OK).



