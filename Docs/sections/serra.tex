As previously discussed, the Prediction Module requires information about the user's storage device to perform scheduling algorithm selection. Additionally, other parts of the library implementation also use this information. This information is provided to AGIOS through two a file generated by SeRRa. The tool and information on how to use it can be obtained at \url{http://serratool.bitbucket.org/}.

The file name (with its path) must be provided to AGIOS through its configuration file. The next chapter will give more details about the configuration file.

\begin{lstlisting}[language=bash]
2
8	64
566.8405	6784.471
238.2483	1173.456
145.6056	1018.143
187.2437	-201.8793
64	128
571.2364	-740.0433
234.6933	41.99119
218.3765	-6295.179
113.1743	2235.442
\end{lstlisting}

We show above an example of the file generated by SeRRa. This file gives estimated access times in the form of linear functions. The first line gives the number of considered intervals, and then intervals are described one after the other, in request size order. Each interval takes $5$ lines, where the first one gives the interval information. In this example, we have two intervals: [$8$KB, $64$KB] and [$64$KB, $128$KB]. The following $4$ lines define the access times functions for different access patterns in the following order: sequential write, random write, sequential read, and random read. In the example, the time to perform sequential write in the profiled storage device is given by \\
$time(request\_size) = 566.8405 \times request\_size + 6784.471 nanoseconds$.

If the library needs to estimate a request size which is not included in the described intervals, it will use the function provided for the last interval of this file. However, to have more accurate results, the following guidelines should be taken into consideration when generating the file with SeRRa:

\begin{itemize}
\item Describe intervals to SeRRa comprising at least requests from $4$KB to $4$MB. We suggest separating these values into two intervals, breaking at $64$KB. In our experience with multiple storage devices, this leads to the best results.
\item If you can afford the profiling time, use as many benchmark repetitions as possible (this is a parameter provided to SeRRa, see its documentation for more detail).
\item If you are using AGIOS on a Grid'5000 cluster, take a look at our page since we provide files obtained at some clusters.
\end{itemize}

The access time estimations are used by the aIOLi scheduling algorithm (to quantum assignment) and by the Prediction Module (to maintain the $\alpha$ factor and test against predicted aggregations). They are also used to the mechanism which automatically selects the best scheduling algorithm to obtain the sequential to random throughput ratio of the storage device. If you are using AGIOS without predicted request aggregations and with another scheduling algorithm (statically, without automatic scheduling algorithm selection), you can provide any values (the example file that comes within the library package is OK).

