AGIOS was developed in the context of the joint laboratory LICIA\footnote{http://licia-lab.org/index-en.html} between the Federal University of Rio Grande do Sul (UFRGS), in Brazil, and the University of Grenoble, in France. The research was supported by CAPES-BRAZIL - under the grant 5847/11-7 - and the HPC-GA international project\footnote{https://project.inria.fr/HPC-GA/}. 

Experiments that allowed its progress were conducted on the Grid'5000 experimental test bed, being
developed under the INRIA ALADDIN development action with support from
CNRS, RENATER and several Universities as well as other funding bodies
(see \url{https://www.grid5000.fr}).

This tool's source code is open and you are allowed to make modifications and adjust to your purposes, as long as your version is also freely available. 

If you have further questions or want to discuss modifications to the tool, please contact Francieli Zanon Boito: fzboito [at] inf [dot] ufrgs [dot] br.

\section{Publications}

If AGIOS is useful to you, consider citing one of our publications in your research work.

\begin{itemize}
\item ``Automatic I/O scheduling algorithm selection for parallel file systems''. In Concurrency and Computation: Practice and Experience, Wiley, 2015. \url{http://onlinelibrary.wiley.com/doi/10.1002/cpe.3606/abstract}
\item ``AGIOS: Application-Guided I/O Scheduling for Parallel File Systems''. In Parallel and Distributed Systems (ICPADS), 2013 International Conference on. IEEE. \url{http://ieeexplore.ieee.org/xpl/articleDetails.jsp?arnumber=6808156}
\end{itemize}

\section{Collaborators}

The following people are (or have been at some point) involved with the development of AGIOS:

\begin{itemize}
\item Philippe O. A. Navaux, Universidade Federal do Rio Grande do Sul (UFRGS), Brazil;
\item Yves Denneulin, Laboratoire d'Informatique de Grenoble (LIG) - INRIA, France.
\end{itemize}
